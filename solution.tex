\documentclass{article}\usepackage[utf8]{inputenc}\title{Relatório de Entregas}\author{Reinaldo J. Dias de Abreu e Mirrális Dias Santana}\date{\today}\begin{document}\maketitle\subsection*{Tabela de Entregas}\begin{table}[!hb]\centering\begin{tabular}{|c|c|c|}\hline Farmácia  & Quantidade & Cliente\\ \hline    1  &       5  &      1 \\    1  &      45  &      2 \\    2  &      50  &      3 \\    3  &      50  &      3 \\\hline\end{tabular}\end{table}\subsubsection*{Custo Total:     14.000}\subsubsection*{Número de Farmácias:          3}\subsubsection*{Número de Clientes:          3}\end{document}